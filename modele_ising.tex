\documentclass[a4paper,12pt]{report}

\usepackage[utf8]{inputenc}
\usepackage[francais]{babel}

\usepackage{graphicx}
%\usepackage{amsthm}
\usepackage{amsmath}
\usepackage{amsfonts}
%\usepackage{dsfonts}
\usepackage{amssymb}

% Encadrer les theoremes
\usepackage{framed} % or, "mdframed"
\usepackage[framed]{ntheorem}
\newframedtheorem{thm}{Théorème}
\newframedtheorem{prop}{Proposition}
\newframedtheorem{defi}{Définition}
\newframedtheorem{lemme}{Lemme}

\usepackage[top=2cm, bottom=2cm, left=2cm, right=2cm]{geometry}

% def==
\newcommand\eqdef{\mathrel{\stackrel{\makebox[0pt]{\mbox{\normalfont\tiny def}}}{=}}}

% Numérotation des sections en chiffres romains
\renewcommand{\thesection}{\Roman{section}.}

% Environnement de preuves
\newenvironment{preuve}{\paragraph{Preuve:}}{\hfill$\square$}


\title{ \begin{Huge} Le modèle d'Ising \end{Huge} }

\begin{document}

\maketitle

\newpage

\tableofcontents

\newpage

\section{Introduction}
Modèle introduit par Lenz en 1920. On considère un graphe $G := (V_G, E_G)$. A chaque sommet de $v \in V_G$ on associe une valeur $\sigma_v \in \{-1, 1\}$. L'énergie d'une telle configuration est alors donnée par:
$$
H_G(\sigma) = - \sum_{(x,y) \in E_G} \sigma_x \sigma_y
$$
soit deux fois le nombre de sommets voisins qui ne sont pas "alignés". Pour $(x,y) \in E_G$, on notera $x \sim y$.

\begin{defi}
La mesure de probabilité du modèle d'Ising sur le graphe $G$ à température inverse $\beta$ est définie par:

	$$
\langle X \rangle_{G, \beta}^{f} = \frac{1}{Z^f_{(G,\beta)}} \sum_{\sigma \in \{\pm 1\}^{V_G}} X(\sigma) \exp(-\beta H_G(\sigma))
	$$
	où $X$ est une fonction $\{\pm 1\}^{V_G} \rightarrow \mathbb{R}$ et où $Z^f_{(G,\beta)}=\sum_{\sigma \in \{\pm 1\}^{V_G}} e^{-\beta H_G(\sigma)}$ est appelée fonction de partition du modèle.
	\end{defi}

Remarques:
\begin{itemize}
\item la mesure est invariante par inversement de tous les signes: "symétrie"
\item on voudrait travailler sur de plus grands graphes, $\mathbb{Z}^d$ par exemple
\end{itemize}

On considère maintenant $G \subset \mathbb{Z}^d$ et on définit la frontière $\partial G$ de $G$ par:
$$
\partial G = \lbrace x \in V_G \vert \exists y \in \mathbb{Z}^d \setminus G, y \sim x \rbrace
$$

Pour rompre la symétrie, on "plonge" $G$ dans un monde $+$ en définissant le nouvel Hamiltonien:
$$
H_G(\sigma)^+ = - \sum_{x \sim y} \sigma_x \sigma_y  -  \sum_{x \in \partial G} \sigma_x
$$

On pose maintenant $\Lambda_n = [-n,n]^d$ et on définit la magnétisation spontanée:
$$
m(\beta) = \inf_{n \in \mathbb{N}} \langle \sigma_0 \rangle^+_{\Lambda_n, \beta}
$$

\begin{defi}
On définit la température inverse critique $\beta_c^{(d)} = \inf \lbrace \beta > 0 \vert m(\beta) > 0 \rbrace$
\end{defi}

Ising a montré que $\beta_c^{(1)} = + \infty$ puis Peirls montre que $\forall d \geq 2, \beta_c^{(d)} \in ]0, + \infty [$.
\newline

\underline{Questions}
\begin{enumerate}
\item $\beta_c^{(d)} = 0$ ?
\item $m(\beta_c^{(d)})=0$ ?
\item $\lim_{\beta \downarrow \beta_c} m(\beta)$ ?
\item comportement de $m(\beta)$ autour de $\beta_c$ ?
\item vitesse de décroissance en $n$ de $\langle \sigma_0 \rangle^+_{\Lambda_n, \beta}$ ?
\item universalité?
\end{enumerate}

\begin{thm}[Kramers - Wanier]
$\beta_c(2) = \frac{1}{2} \log(1 + \sqrt{2})$
\end{thm}

Pour $d \geq 3$ on n'a pas de formule explicite pour $\beta_c(d)$.

\begin{thm}[Aizenman - Barsky - Fernandez (1980)]
$$\forall \beta < \beta_c, \langle \sigma_0 \rangle^+_{\Lambda_n, \beta} \leq e^{-cn}$$
où $c>0$ est une constante dépendant de $\beta$.
\end{thm}

\underline{Conjecture:} $m(\beta) = (\beta - \beta_c)^{\overline{\beta} + o(1)}$ quand $\beta \downarrow \beta_c$, avec
\begin{equation*}
\overline{\beta} =
\begin{cases}
1/8 & \text{si } d=2 \text{ (prouvé)}\\
? & \text{si } d=3 \\
1/2 & \text{si } d \geq 4 \text{ (prouvé)}
\end{cases}
\end{equation*}

Pour $d=3, m(\beta_c)=0$ (Aizenman, DC, 2015).
\newline

\underline{Conjecture:} $\langle \sigma_0 \rangle^+_{\Lambda_n, \beta_c} = \frac{1}{n^{\alpha_1 + o(1)}}$
\newline
\begin{equation*}
\alpha_1 =
\begin{cases}
1/8 & \text{si } d=2 \\
? & \text{si } d=3 \\
(d-2)/2 & \text{si } d \geq 4
\end{cases}
\end{equation*}
\newline
La corrélation (ici entre l'origine et la frontière) décroit donc à vitesse $n^{-\alpha_1}$ où $n$ est la distance entre l'origine et la frontière.
\newline
\newline
Au lieu de faire tendre les limites du graphe vers l'infini, on peut aussi faire tendre la maille dur réseau vers 0. On se donne $\Omega$ un ouvert de $\mathbb{R}^d$. On se donne le graphe $\Omega_{\delta} = \Omega \cap \delta \mathbb{Z}^d$. Pour $x \in \Omega$, on pose $x^{\delta}$ un point de $\Omega_{\delta}$ très proche de $x$. On définit pour $x_1, \dots, x_k \in \Omega$:
$$
\langle \sigma_{x_1} \dots \sigma_{x_k} \rangle_{\Omega} = \lim_{\delta \downarrow 0} \frac{\langle \sigma_{x_1^{\delta}} \dots \sigma_{x_k^{\delta}} \rangle_{\Omega_{\delta}}}{\delta^{k\alpha_1}}
$$
\newline

\underline{Universalité:}
L'universalité signifie que les résultats restent valables si on transforme le graphe. C'est encore une question très ouverte.
On a une relation entre la corrélation d'un nombre paire de spins et celle de deux spins par le théorème suivant:

\begin{thm}[Aizenman]
$$\langle \sigma_{x_1} \dots \sigma_{x_{2k}} \rangle = \sum_{\pi \text{ pairing}} \langle \sigma_{x_{\pi(1)}} \sigma_{x_{\pi(2)}} \rangle \dots \langle \sigma_{x_{\pi(2k-1)}}\sigma_{x_{\pi(2k)}} \rangle$$
\end{thm}

On considère une transformation conforme $\phi$ dans $\mathbb{R}^2$ (holomorphe et bijective) entre $\Omega$ et $\phi(\Omega)$, on a un résultat pour l'universalité:

\begin{thm}[Chelkak-Hongler-Izyurov]
$$\langle \sigma_{x_1} \dots \sigma_{x_k} \rangle_{\Omega} = |\phi'(x_1)|^{\frac{1}{8}}\dots |\phi'(x_k)|^{\frac{1}{8}} \langle \sigma_{\phi(x_1)} \dots \sigma_{\phi(x_k)} \rangle_{\phi(\Omega)}$$
\end{thm}

\newpage
\section{Représentation en courant aléatoire}
\subsection{Définitions}
\underline{But:} Obtenir/étudier une représentation gémotétrique du modèle d'Ising

\begin{defi}[Courant]
Un courant sur $G$ est une application: 
$$n :\begin{array}[t]{rcl} E_G \rightarrow \mathbb{N} \\ xy \mapsto n_{xy} \end{array}$$
\end{defi}

Etant donné un courant $n$, on définit l'ensemble $\partial n$, l'ensemble des sources de $n$, par:
$$
\partial n = \Big\lbrace x \in V_G \ \vert \ \sum_{y \sim x} n_{xy} \text{ impair} \Big\rbrace
$$
On définit le poids de $n$ par:
$$
w_{\beta}(n) = \prod_{x \sim y} \frac{\beta^{n_{xy}}}{n_{xy}!}
$$

Pour $A \subset V_G$ on définit une probabilité sur l'ensemble des chemins par:
$$
P^A_{\beta}(n) \propto 
\begin{cases}
0 & \text{si } \partial n \neq A \\
w_{\beta}(n) & \text{si } \partial n = A \\
\end{cases}
$$
Cette probabilité ne met du poids que sur les chemins dont les sources sont exactement $A$.
Soit $A \subset V_G$, on définit:
$$
Z_{G,\beta}^f(A) \ \eqdef \ \sum_{\sigma \in \{ \pm 1 \}^{V_G} } \left( \prod_{x \in A} \sigma_x \right) e^{- \beta H_G(\sigma)}
$$
or
$$
\exp(-\beta H_G(\sigma)) = \prod_{x \sim y} \exp(\beta \sigma_x \sigma_y) = \prod_{x \sim y} \sum_{n_{xy}=0}^{+ \infty} \frac{(\beta \sigma_x \sigma_y)^{n_{xy}}}{n_{xy}!}
$$
d'où:
$$
Z_{G,\beta}^f(A) = \sum_{n \text{ courant}} \left( \prod_{x \sim y} \frac{\beta^{n_{xy}}}{n_{xy}!} \right) \left( \sum_{\sigma \in \{ \pm 1 \}^{V_G}} \prod_{x \in V_G} \sigma_x^{\sum_{x \sim y} n_{xy} + \delta_{x \in A}} \right)
$$
Beaucoup de termes de la somme principale sont nuls. Si $n$ est un courant tel que $\partial n \neq A$, alors le produit $\prod_{x \in V_G} \sigma_x^{\sum_{x \sim y} n_{xy} + \delta_{x \in A}}$ contient un terme $\sigma_x$ élevé à une puissance impaire. Cela implique que la somme de ces produits pour $\sigma \in \{ \pm 1 \}^{V_G}$ s'annule.
\newline

Il ne reste donc que:
\begin{prop}
$$
Z_{G,\beta}^f(A) = 2^{\vert V_G \vert} \sum_{\partial n  = A} w_{\beta}(n)
$$
\end{prop}

On en déduit $\langle \prod_{x \in A} \sigma_x \rangle_{G,\beta}^f$, la constante de normalisation étant obtenue pour $A = \emptyset$:
$$
\langle \prod_{x \in A} \sigma_x \rangle_{G,\beta}^f = \frac{\sum_{\partial n = A} w_{\beta}(n)}{\sum_{\partial n = \emptyset} w_{\beta}(n)}
$$

\begin{prop}[Première inégalité de Griffiths]
$$
\forall A \subset V_G, \ \langle \prod_{x \in A} \sigma_x \rangle_{G,\beta}^f \geq 0
$$
\end{prop}

\underline{Cas de l'environnement +:}
Travailler avec le Hamiltonien $H_{G}^+$ revient rajouter un sommet fantôme $g$ à $G$ et le relier à tous les sommets de $\partial G$. Si on impose $\sigma_g = +1$, le Hamiltonien est exactement $H_{G}^+$. On peut alors refaire le même calcul de $Z_{G,\beta}^+(A)$, pour le graphe $G^g$ auquel on a rajouté $g$. On a alors:
$$
Z_{G,\beta}^+(A) = 2^{\vert V_G \vert} \sum_{\partial n \cap V_G  = A} w_{\beta}(n)
$$

\underline{Décomposition d'un courant:} On peut décomposer un courant en boucles et en chemins reliant deux sources. Etant donné un courant sans sources on peut construire un chemin sur $G$ partant d'un sommet $x$ de "degré" $\sum_{y \sim x} n_{xy}$ non nul (et pair) en se déplaçant sur une arrète $xy$ telle que $n_{xy}>0$. On retire 1 à cette arrète et on continu le chemin à partir de $y$ (ce qui est forcément possible vu que $y$ a maintenant un degré impair). L'opération se termine lorsque l'on revient en $x$. On décompose donc un courant sans sources en boucles. Si le courant a des boucles, on construit de la même façon des chemins qui partent des sources (et terminent donc sur ces sources), jusqu'à ce que le courant n'aie plus de sources. On le décompose alors en boucles. On va interprèter ces chemins en termes de marches aléatoires sur $G$\dots

\subsection{Le ``switching lemma"}

On commence par l'énoncé du lemme.
\begin{thm}[Switching Lemma]
Soit $A\subset G, x,y \in V_G, \forall F: \mathbb{N}^{E_G} \rightarrow \mathbb{R}$,
$$\sum_{\overset{\partial n_1 = A,}{\partial n_2 = \{x,y\}}} F(n_1+n_2)w_\beta(n_1)w_\beta(n_2) = \sum_{\overset{\partial n_1 = A \Delta\{x,y\},}{\partial n_2 = \emptyset}}F(n_1+n_2)w_\beta(n_1)w_\beta(n_2)\mathbf{1}(x\overset{\widehat{n_1+n_2}}{\longleftrightarrow}y)
$$
\end{thm}
On dit définit un modèle de percolation induit par un courant $\hat{n}_e = \mathbf{1}(n_e>0)$.
\begin{preuve}
On note $n = n_2, m = n_1+n_2$.
$$w_\beta(n_1)w_\beta(n_2) = \prod_{e}\frac{\beta^{(n_1+n_2)_e}}{(n_1)_e!(n_2)_e!} = w_\beta(m) \prod_e\left(\begin{array}{c} m_e \\ n_e \end{array} \right)
$$
On note $\left(\begin{array}{c} m \\ n \end{array} \right)$ le nombre de façons d'extraire un sous courant $n$ depuis un courant $m$, on a $\prod_e\left(\begin{array}{c} m_e \\ n_e \end{array} \right) = \left(\begin{array}{c} m \\ n \end{array} \right)$.

Le terme à gauche devient:
$$
\sum_{\begin{array}{c}
\partial m = A\Delta\{x,y\},\\
n\leqslant m, \\
\partial n = \emptyset
\end{array}} F(m)w_\beta(m) \left(\begin{array}{c} m \\ n
\end{array} \right)= \sum_{\partial m = A\Delta\{x,y\}}F(m)w_\beta(m)\sum_{\overset{n\leqslant m,}{\partial n = \{x,y\}}}\left(\begin{array}{c} m \\ n
\end{array} \right)
$$
La seconde somme est en fait $\#\big\{ \mathcal{N}\subset \mathcal{M},\partial \mathcal{N} = \{x,y\}\big\}$.

Le terme à droite est en fait $$
\sum_{\partial m = A \Delta \{x,y\}} F(m)w_\beta(m)\mathbf{1}(x\overset{\hat{m}}{\longleftrightarrow}y)\sum_{\overset{\partial n = \emptyset}{n \leqslant m}}\left(\begin{array}{c} m \\ n \end{array} \right).
$$
On a $\displaystyle\sum_{\overset{\partial n = \emptyset}{n \leqslant m}}\left(\begin{array}{c} m \\ n \end{array} \right) = \#\{\mathcal{N}\subset \mathcal{M}, \partial \mathcal{N} = \emptyset\}$.
Il reste à voir que  $$\#\big\{ \mathcal{N}\subset \mathcal{M},\partial \mathcal{N} = \{x,y\}\big\} = \mathbf{1}(x\overset{\hat{m}}{\longleftrightarrow}y)\#\{\mathcal{N}\subset \mathcal{M}, \partial \mathcal{N} = \emptyset\}.$$
Le cas où $x$ non connecté à $y$ est évident, pour le cas $x\overset{\hat{m}}{\leftrightarrow}y$ et on observe que $\mathcal{N} \longleftrightarrow \mathcal{N}\Delta\mathcal{K}$ est bijective entre les deux ensembles pour un $\mathcal{K}$ tel que $\partial \mathcal{K} = \{x,y\}$.
\end{preuve}

\begin{prop}[Corollaires du switching lemma]\leavevmode
\begin{enumerate}
\item $(\langle \sigma_x \sigma_y \rangle_{G,\beta}^f)^2 = P_{G,\beta}^\emptyset \otimes P_{G,\beta}^\emptyset (x\overset{\widehat{n_1+n_2}}{\longleftrightarrow} y)$
\item $(\langle \sigma_0\rangle_{G,\beta}^+)^2= P_{G\cup\{g\},\beta}^\emptyset \otimes P_{G\cup\{g\},\beta}^\emptyset (0\overset{\widehat{n_1+n_2}}{\longleftrightarrow} g)$
\item $\displaystyle \frac{d}{d\beta}\langle \sigma_x \sigma_y \rangle_{G,\beta}^f \geqslant 0, \frac{d}{d\beta}\langle \sigma_0 \rangle_{G,\beta}^+ \geqslant 0$
\end{enumerate}
\end{prop}

\begin{preuve}

\begin{enumerate}

\item 
\begin{align*}
(\langle \sigma_x \sigma_y \rangle_{G,\beta}^f)^2 
&= \frac{Z^2(\{x,y\})}{Z^2(\emptyset)} 
= \frac{\sum_{\partial n_1 = \{x,y\},\partial n_2 = \{x,y\}}w_\beta(n_1)w_\beta(n_2)}{\sum_{\partial n_1 = \emptyset,\partial n_2 = \emptyset}w_\beta(n_1)w_\beta(n_2)}\\ 
&\overset{SL}{=} \frac{\sum_{\partial n_1 = \partial n_2 = \emptyset}w_\beta(n_1)w_\beta(n_2)}{\sum_{\partial n_1 = \partial n_2 = \emptyset}w_\beta(n_1)w_\beta(n_2)} = P_{G,\beta}^\emptyset \otimes P_{G,\beta}^\emptyset (x\overset{\widehat{n_1+n_2}}{\longleftrightarrow} y)
\end{align*}

\item idem en ajoutant le point fantôme.

\item 
\begin{align*}
&\frac{d}{d\beta}\langle \sigma_x \sigma_y \rangle_{G,\beta}^f 
= \frac{d}{d\beta} \frac{\sum \sigma_x \sigma_y e^{\beta \sum_{x\sim y}\sigma_u \sigma_v}}{\sum e^{\beta\sum_{u\sim v}\sigma_u \sigma_v}}\\ 
&= \frac{(\sum_{x\sim y,u\sim v}\sigma_x\sigma_y\sigma_u\sigma_v e^{\beta \sum \sigma_u\sigma_v})(\sum e^{\beta\sum_{u\sim v}\sigma_u \sigma_v}) - (\sum \sigma_x\sigma_y e^{\beta\sum\sigma_u\sigma_v})(\sum\sum_{u\sim v}\sigma_u\sigma_v e^{\beta \sum\sigma_u\sigma_v})}{(\sum e^{\beta \sum \sigma_u \sigma_v})^2} \\
&= \sum_{u\sim v}\langle \sigma_x \sigma_y \sigma_u \sigma_v\rangle - \langle \sigma_x \sigma_y\rangle\langle \sigma_u \sigma_v \rangle \\
\end{align*}
\begin{align*}
\langle \sigma_x \sigma_y \sigma_u \sigma_v\rangle - \langle \sigma_x \sigma_y\rangle\langle \sigma_u \sigma_v \rangle 
&= \frac{Z(\{x,y,u,v\})Z(\emptyset) - Z(\{x,y\})Z(\{u,v\})}{Z^2(\emptyset)} \\ 
&= \frac{\sum_{\partial n_1 = \{x,y\}\Delta\{u,v\},\partial n_2 = \emptyset}w(n_1)w(n_2)(1-\mathbf{1}(u\longleftrightarrow v))}{Z^2(\emptyset)} \geq 0
\end{align*}
\end{enumerate}
\end{preuve}

\begin{prop}[Deuxième inégalité de Griffiths]\
Pour tout ensembles de spins $A$ et $x,y \in V_G$:
$$
\langle (\prod_{a \in A} \sigma_a) \sigma_x \sigma_y \rangle \geq \langle \prod_{a \in A} \sigma_a \rangle \langle \sigma_x \sigma_y \rangle
$$
\end{prop}

\begin{preuve}
\begin{align*}
\langle \prod_{a \in A} \sigma_a \rangle \langle \sigma_x \sigma_y \rangle
&= \frac{Z(A) Z(B)}{Z(\emptyset)^2} 
= \frac{\sum_{\overset{\partial n_1 = A}{\partial n_2 = \{x,y\} }} w(n_1) w(n_2) }{Z(\emptyset)^2}
\overset{SL}{=} \frac{\sum_{\overset{\partial n_1 = A \Delta \{x,y\}}{\partial n_2 = \emptyset}} w(n_1) w(n_2) \mathbf{1}_{x \leftrightarrow y}}{Z(\emptyset)^2} \\
&\leq \frac{\sum_{\overset{\partial n_1 = A \Delta \{x,y\}}{\partial n_2 = \emptyset}} w(n_1) w(n_2)}{Z(\emptyset)^2} \\
&= \frac{\sum_{\partial n_1 = A \Delta \{x,y\}} w(n_1) w(n_2)}{Z(\emptyset)} = \langle (\prod_{a \in A} \sigma_a) \sigma_x \sigma_y \rangle 
\end{align*}
\end{preuve}


Enfin une inégalité qui joue le rôle de l'inégalité de BK de la percolation de Bernoulli:

\begin{prop}[Inégalité de Simon-Lieb]
Soit $S\subset \mathbb{Z}^d$ séparant $0$ de $x$.
$$\langle \sigma_0 \sigma_x \rangle_{G,\beta}^f \leqslant \sum_{y\in S}\langle \sigma_0 \sigma_y\rangle_{G,\beta}^f \langle \sigma_y \sigma_x\rangle_{G,\beta}^f
$$
\end{prop}

\begin{preuve}
$$\frac{\langle \sigma_0 \sigma_y \rangle\langle\sigma_y \sigma_x\rangle}{\langle \sigma_0 \sigma_x\rangle} = \frac{Z(\{0,y\})Z(\{y,x\})}{Z(\{0,x\})Z(\emptyset)} = \frac{\sum_{\partial n_1=\{0,x\},\partial n_2= \emptyset} w(n_1)w(n_2)\mathbf{1}(x\leftrightarrow y)}{\sum_{\partial n_1 = \{0,x\},\partial n_2 = \emptyset}w(n_1)w(n_2)} = P^{\{0,x\}}\otimes P^\emptyset(x\leftrightarrow y)
$$
On somme les $y\in S$ et on a 
$$\sum_{y\in S}\frac{\langle \sigma_0 \sigma_y \rangle\langle\sigma_y \sigma_x\rangle}{\langle \sigma_0 \sigma_x\rangle} = \sum_{y\in S}P^{\{0,x\}}\otimes P^\emptyset(x\leftrightarrow y) \geqslant 1.
$$
\end{preuve}
\newpage

\section{Comportement pour $\beta < \beta_c$}

On définit, pour S un ensemble finit de sommets: $\tilde{\phi}_{\beta}(S) = \sum_{x \in \partial S} \langle \sigma_0 \sigma_x \rangle_{\mathbb{Z}^d,\beta}^f$

\begin{lemme}
Supposons qu'il existe $S \subset \mathbb{Z}^d$ fini, contenant $0$, et tel que $\tilde{\phi}_{\beta}(S)<1$. Il existe alors $c > 0$ tel que:
$$
\forall x \in \mathbb{Z}^d, \ \langle \sigma_0 \sigma_x \rangle_{\mathbb{Z}^d,\beta}^f \leq e^{-c\vert x \vert}
$$
\end{lemme}

\begin{preuve}
Par l'inégalité de Simon-Lieb et l'invariance par translation: 
$$\langle \sigma_0 \sigma_x \rangle_{\mathbb{Z}^d,\beta}^f 
\leq \tilde{\phi}_{\beta}(S) \sup_{y \in S} \langle \sigma_y \sigma_x \rangle_{\mathbb{Z}^d,\beta}^f 
= \tilde{\phi}_{\beta}(S) \sup_{y \in S} \langle \sigma_0 \sigma_{x-y} \rangle_{\mathbb{Z}^d,\beta}^f$$

On fixe $n \in \mathbb{N}$ tel que $S \subset \Lambda_n$. Soit $k \in \mathbb{N}^*$:
$$
\sup_{x \notin \Lambda_{kn}} \langle \sigma_0 \sigma_x \rangle_{\mathbb{Z}^d,\beta}^f 
\leq \tilde{\phi}_{\beta}(S) \sup_{x \notin \Lambda_{kn}} \sup_{y \in S} \langle \sigma_0 \sigma_{x-y} \rangle_{\mathbb{Z}^d,\beta}^f
\leq \tilde{\phi}_{\beta}(S) \sup_{x \notin \Lambda_{(k-1)n}} \langle \sigma_0 \sigma_{x} \rangle_{\mathbb{Z}^d,\beta}^f$$

Car pour $y \in S \subset \Lambda_n$ et $x \notin \Lambda_{kn}$, $x-y \notin \Lambda_{(k-1)n}$. En itérant la relation et en utilisant que $\tilde{\phi}_{\beta}(S)<1$, on obtient la décroissance exponentielle.
\end{preuve}

On peut également définir
$$\phi_{\beta}(S) = tanh(\beta) \sum_{\overset{x\in S, y \notin S}{x \sim y}} \langle \sigma_0 \sigma_x \rangle^f_{S,\beta}$$
et montrer de même qu'il y a décroissance exponantielle dès que $\phi_{\beta}(S) <1$. Cette fois, le critère porte uniquement sur la mesure finie du modèle d'Ising sur S (fini). $\phi_{\beta}$ est donc continue. Cela implique que ce critère de décroissance exponentielle ne s'applique pas au point critique $\beta_c$ sinon on aurait également décroissance exponentielle pour un $\beta > \beta_c$. Au point critique, on a aussi pour tout $n \in \mathbb{N}^*, \ \phi_{\beta_c}(\Lambda_n) \geq 1$. En sommant sur $n$, on obtient:
$$
\sum_{x \in \mathbb{Z}^d} \langle \sigma_0 \sigma_x \rangle^f_{\beta_c, \mathbb{Z}^d} = + \infty
$$

\begin{thm}[Aizenman, Barsky, Fernandez]
Soit $d \geq 2$,
\begin{itemize}
\item Si $\beta < \beta_c$, alors $\exists c >0, \ \forall x \in \mathbb{Z}^d, \ \langle \sigma_0 \sigma_x \rangle \leq e^{-c \vert x \vert}$
\item Si $\beta > \beta_c$, alors $\langle \sigma_0 \rangle^+_{\mathbb{Z}^d,\beta} \geq \sqrt{\frac{\beta^2 - \beta_c^2}{\beta^2}}$
\end{itemize}
\end{thm}
\newpage

\section{Le modèle d'Ising en dimension $d \geq 5$}

On considère 4 points de $\mathbb{Z}^d$, notés 1, 2, 3 et 4. On pose:
$$
U_4^{\beta} = \langle \sigma_1 \sigma_2 \sigma_3 \sigma_4 \rangle - \langle \sigma_1 \sigma_2 \rangle \langle \sigma_3 \sigma_4 \rangle - \langle \sigma_1 \sigma_3 \rangle \langle \sigma_2 \sigma_4 \rangle - \langle \sigma_1 \sigma_4 \rangle \langle \sigma_2 \sigma_3 \rangle
$$

\begin{prop}
$$
U_4^{\beta} = -2\langle \sigma_1 \sigma_2 \rangle \langle \sigma_3 \sigma_4 \rangle P^{\{1,2\}} \otimes P^{\{3,4\}}(\text{1,2,3,4 reliés})
$$
\end{prop}

\begin{preuve}

\begin{align*}
U_4^{\beta} 
&= \frac{Z(1,2,3,4)Z(\emptyset) - Z(1,2)Z(3,4) - Z(1,3)Z(2,4) - Z(1,4)Z(2,3)}{Z(\emptyset)^2} \\
&\overset{SL}{=} \frac{1}{Z(\emptyset)^2} \sum_{\overset{\partial n_1 = \{1,2,3,4\}}{\partial n_2 =\emptyset}} w(n_1) w(n_2) (1 - \mathbf{1}_{3 \leftrightarrow 4} - \mathbf{1}_{2 \leftrightarrow 4} - \mathbf{1}_{2 \leftrightarrow 3}) \\
&= \frac{1}{Z(\emptyset)^2} \sum_{\overset{\partial n_1 = \{1,2,3,4\}}{\partial n_2 =\emptyset}} w(n_1) w(n_2) (-2) \mathbf{1}_{1,2,3,4 \text{ reliés}} \\
&\overset{SL}{=} \frac{-2}{Z(\emptyset)^2} \sum_{\overset{\partial n_1 = \{1,2\}}{\partial n_2 =\{3,4\}}} w(n_1) w(n_2) \mathbf{1}_{1,2,3,4 \text{ reliés}} \\
&= \frac{-2 Z(1,2) Z(3,4)}{Z(\emptyset)^2}
\frac{ \sum_{\overset{\partial n_1 = \{1,2\}}{\partial n_2 =\{3,4\}}} w(n_1) w(n_2) \mathbf{1}_{1,2,3,4 \text{ reliés}}}{Z(1,2) Z(3,4)} \\
&= -2\langle \sigma_1 \sigma_2 \rangle \langle \sigma_3 \sigma_4 \rangle P^{\{1,2\}} \otimes P^{\{3,4\}}(\text{1,2,3,4 reliés})
\end{align*}

\end{preuve}

\begin{thm}[Infrared bound]
$\exists c > 0, \ \forall \beta \leq \beta_c,$
$$
\langle \sigma_0, \sigma_x \rangle^f_{\beta} \leq \frac{c}{\beta} G(0,x) \leq \frac{\tilde{c}}{\beta \vert x \vert^{d-2}}
$$
\end{thm}

\begin{prop}
$$
0 \leq - U_4(1,2,3,4) \leq 2 \sum_{y \in \mathbb{Z}^d} \langle \sigma_1 \sigma_y \rangle \langle \sigma_2 \sigma_y \rangle \langle \sigma_3 \sigma_y \rangle \langle \sigma_4 \sigma_y \rangle
$$
\end{prop}

On se place à $\beta < \beta_c$. On définit la susceptibilité
$$
\chi(\beta) = \sum_{x \in \mathbb{Z}^d} \langle \sigma_0 \sigma_x \rangle
$$
ainsi que la longueur de corrélation
$$
\xi_{\beta} = \big( \lim_{n \rightarrow + \infty} - \frac{1}{n} log( \langle \sigma_0 \sigma_{n e_1} \rangle ) \big)^{-1}
$$
Montrons que $xi_{\beta}$ est bien défini. On pose $a_n = - log( \langle \sigma_0 \sigma_{n e_1} \rangle )$. Par le théorème de décroissance exponentielle, on sait qu'il existe $c>0$ tel que $a_n \geq c n$. Par ailleurs
$$
\langle \sigma_0 \sigma_{(n+m)e_1} \rangle = \langle \sigma_0 \sigma_{ne_1} \sigma_{ne_1} \sigma_{(n+m)e_1} \rangle 
\geq \langle \sigma_0 \sigma_{ne_1} \rangle \langle \sigma_{ne_1} \sigma{(n+m)e_1} \rangle 
= \langle \sigma_0 \sigma_{ne_1} \rangle \langle \sigma_0 \sigma_{me_1} \rangle
$$
Où l'on a appliqué successivement la deuxième inégalité de Griffiths et l'invariance par translation du modèle. On a donc:
$$
a_{m+n} \leq a_n + a_m
$$
$(a_n)$ est donc sous-additive, le lemme de Fekete garanti que $\frac{a_n}{n} \rightarrow l \geq c > 0$. $\xi_{\beta}$ est donc bien défini. Par ailleurs, le lemme de Fekete donne aussi que $l = \inf \{ a_n / n \vert n \in \mathbb{N}^* \}$. On a donc que
$$
\langle \sigma_0 \sigma_{ne_1} \rangle_{\beta}^f \leq \exp(- \frac{n}{\xi_{\beta}})
$$
On peut généraliser cette inégalité. Soit $x \in \mathbb{Z}^d$,
$$
\langle \sigma_0 \sigma_x \rangle \langle \sigma_x \sigma_{2 \Vert x \Vert_{\infty} e_1} \rangle \leq \langle \sigma_0 \sigma_{2 \Vert x \Vert_{\infty}} \rangle \leq \exp(-\frac{2 \Vert x \Vert_{\infty}}{\xi_{\beta}})
$$
d'où
$$
\langle \sigma_0 \sigma_x \rangle_{\beta}^f \leq \exp(-\frac{\Vert x \Vert_{\infty}}{\xi_{\beta}})
$$
\end{document}
